\chapter{結論}

\section{少量データに対する解析手法}
2次元画像を入力にして,比較したのは,データ拡張,事前学習,擬似HE染色の3つの処理の有効性について,および2次元の画像の特徴抽出に利用されるモデルの選択について実験,議論した.

検体のサンプル数が少ない状況でかつ,サンプル間のデータのばらつきが大きい場合は,普通の学習方法では,過学習してしまい未知のサンプルの予測精度は大きく減少してしまう.それを防ぐための手法として,データ拡張によってデータの数を擬似的に増やすことで学習データのバリエーションを増やすことや,擬似HE染色による前処理によって元の顕微鏡撮影像よりも情報が落ちてしまうが,サンプル間でのデータのばらつきが少なくなるというメリットがあることが分かった.さらには,世の中の一般的手法はHE染色であるから,オープンソースとなっている画像が多く存在する.そのため事前学習をすることができるようになり,精度を高めることができた.

\section{3次元画像}
3次元画像としてMRIやCTの画像の解析例は多くあり,3DCNNを利用している場合がほとんどである.それは,MRIやCTの深さ方向の解像度が低いことが理由の1つである.しかし今回は深さ方向の解像度が3 $\upmu$m程度であるため,深さ方向に解析するとなると学習のパラメータが大きくなりすぎて学習が進まないという問題があった.これを解決するために動画解析で用いているような時系列解析を3次元画像にも適用した.これにより,2次元の畳み込みに加えて,時系列方向の変化を捉えることができるようになった.

\section{半教師あり学習}
教師ラベルが少ない場合は,教師データに対して過学習することが分かっている.これは学習したサンプル以外のデータを解析すると全く精度が出ないということであり,新しい検体に対して診断システムを利用できなくなってしまう.しかし,腫瘍と正常の形をよく見ると,腫瘍は正常だったものが乱れたり崩れていくことが分かる.正常に対して腫瘍は,異常検知をすることと同義と考え,教師ラベルを貼らなくても,画像の特徴を学習することが可能であることに注目した.全くの教師ラベルなしでも特徴は捉えられるが,そこで少量の教師データを入力することで,どれが正常で,どれが腫瘍かの認識をインストラクションするので,認識精度を向上させることができた.

\section{今後の展望}
最近の動画解析の研究で,ConvLSTMという手法が提案された\cite{要出典}.これは,3DCNNがパラメータが大きくて学習が安定しないという問題と,LSTMだけでは空間的な相互作用が捉えきれないという問題を解決するために,LSTMの内部にCNNを入れている.この手法で本検体を解析することで,さらに3次元構造の特徴を良く捉えることができるか検討する必要がある.

また,GANは生成モデルとしては成功しているが,特徴抽出としてはVAEの方が良いと言われているため,本研究ではVAEを利用した.VAEの学習ではサンプルのデータの分布が正規分布に従うという過程と,AutoEncoderのように元画像に再生成できるかどうかが誤差関数として導入されている.これにGANの仕組みを組み合わせて,VAEで生成される画像がなるべく現実に近い画像となるようにGANを用いるVAE-GANという手法が提案されており,これにより特徴抽出の精度を向上させることが期待される\cite{要出典}.

半教師あり学習では,教師ラベルがない画像と,教師ラベルのある画像を同時に使って学習することができる.教師ラベルとして,人間でも判断が迷うようなものや,医師ごとに見解の異なるラベルなどが混在する場合がある.さらに,医師によってもベテランと研修医では精度に差が生じてしまう.そのため,教師ラベルがあれば高い精度が出るというわけではない.そこで,半教師あり学習のラベルの信頼度を入力に追加することができれば,信頼度によって教師なし学習と教師あり学習の寄与率を変化されることができるはずである.