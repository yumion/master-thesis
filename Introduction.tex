\chapter{序論}
\label{chap_intro}

本研究の目的は内視鏡生検を透明にして深層学習を利用して癌を見落とさない診断方法を開発することである.組織透明化技術LUCIDを用いて検体を丸ごと透明化し,レーザー顕微鏡で観察することで,顕微鏡の解像度で検体の内部まで3次元情報として解析することができる.

透明化されたサンプルをレーザー顕微鏡で撮影すると,従来のカットする手法に換算して1000カットに相当するため,得られる顕微鏡撮影像が従来の2,3カットに対して数100倍となるので,専門医が診断をするには負担が大きくなってしまう.したがって,人工知能(Artificial Intelligence: AI)が3次元画像を解析して病変を検知し,専門医に提示することで病変の見落としリスクがゼロになる診断支援システムを開発することが本研究の目的である.

深層学習で解析を行う場合はデータを大量に準備する必要がある.今回のように,新しい撮影手法であったり,希少な病気であったりすると医療データを数多く集められないことがある.さらに画像データだけでなく,その画像に教師ラベルを貼る(アノテーション)必要がある.画像と,その教師ラベルのセットを数万枚と集めることができたら高い精度が出るという報告が多くあるが,少ない教師データで識別精度を上げることは深層学習において困難とされている.教師データの作成が困難であるだけでなく,少量データである場合は,その教師データが医師によってばらつきがある場合は,教師データを作成した医師の判断が大きく反映されてしまい,必ずしも真であると断言することができなくなってしまう.これらの課題があるため医療画像における深層学習を用いた診断システムは,大量にデータを用意するまでの時間や人的コストを大きく払うことになっているのが現状である.

本研究でこの課題に取り組んだ.教師ラベルを必要とせず,画像の構造的な特徴のパターンを学ぶ教師なし学習の手法と,これまでの教師ラベルを使った学習とを組み合わせた,半教師あり学習(弱教師あり学習)を行った.構造的なパターンは3次元画像であることを活用し,教師なし学習については,病理画像が正常から異常になる過程には連続的な性質があることを活用して正常と異常の分布が作れるようにした.
