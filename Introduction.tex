\chapter{序論}
\label{chap_intro}

\section{研究背景}
ロボティクス技術の向上とともに,様々な分野でロボットが活躍している.特に,医療や福祉といった人間の生活環境下で作業可能なパーソナルロボットの需要の増加は顕著である.しかし,人の生活環境下において完全に自律して動作及び人の補助ができるロボットの実現は困難な状況にある.なぜならば,ロボットが自律行動を行うためには物体や人物の認識,自己位置の認識,そして認識に基づく判断を人の生活環境下で行わなければならないからである.

パーソナルロボットとしてMobile manipulatorの開発が行われている.Mobile Manipulatorは工場に限定されて使用されているが,これを家庭内でも使用できるように軽く,そして小さくすることで衝突リスクを削減する試みがされている.2018年には,Preferred Networksは室内に散らかった家庭用品を片付けるロボットシステムを発表した\cite{お片づけロボット}.部屋の全自動片付けは従来のロボットシステムでは実現困難であったが,近年の深層学習の発展によって初めて実用的なレベルとなった.物をつかむ,物を置く,動作計画を立てる,人の指示に対応するなど,ロボットが人間の生活空間で仕事をするために必要な物体認識・ロボット制御・音声言語理解技術に最先端の深層学習を用いた結果,ロボットが高速・高精度に動作できるようになった.

一方,医療分野におけるロボットで一番身近なものと言えば義手である.義手には装飾用義手や能動義手,作業用義手などがあるが,現在主流の筋電電動義手(筋電義手)は,使用者が筋肉を動かすことで筋電位を読み取り直感的な操作を可能にする.しかし,筋電位を用いた義手には訓練が必要でリハビリテーション施設で行う必要があるが,その施設が極めて少ない\cite{リハビリテーション}.また筋電位が上手く出せない患者や,腕そのものはあるが麻痺して動かせない患者に対しては使用できない.さらに非侵襲的な筋電義手では入力信号が限られる.自由度の高い動作を可能にすると重量が大きくなり,使用者の負担が大きくなってしまう.
実際に筆者が茨城県立大学付属病院リハビリテーション科に見学に行った際,義手使用患者へヒアリングをした.
当患者は普段は能動義手を使用しており,リハビリの際に筋電義手も使用している.半年のリハビリでペンを掴むことまでできたという状況である.
筋電義手の使い心地について聞いたところ,「筋電義手は重い」「(筋電が)ちゃんと伝わっているかわからない」「把持力の制御が難しく,落とすこともよくある」とあまりポジティブな感想はなかった.
以上のような課題があるため装着者のニーズを満たす義手がなく,あまり普及していないのが現状である.


\section{関連研究}
ここではパーソナルロボット開発の先行研究について紹介する.

TOYOTAはHuman Support Robot(HSR)を開発した\cite{HSR}.

机に置いて使うもの
Basic research of upper limb work support system “My Cybernic Robot Arm” for hemiplegic persons


強化学習で制御
教師ありで制御

推論時の計算リソースが必要となる.
cloudリソースを使用する方法がある.これはcloudとのインターフェースを有しているデバイスであれば演算性能は低くても推論が可能となり,スマートフォンやノートPC,ラズパイと言った小さなデバイスを使用することができる.しかし常時インターネットに接続している必要があるため,災害時に使用できなくなるという問題がある.
そこで,cloudではなくlocalで(Edgeで)推論を行うエッジデバイスを使用することがある.

\section{研究目的}
携帯できる生活支援ロボットの開発
タスク:机の上に限定した,ピックアップ
今回はlocalで(Edgeで)処理することにする


