\chapter{序論}
\label{chap_intro}

本研究の目的は、内視鏡生検を透明にして人工知能で癌を見落とさない診断方法を開発することである。組織透明化技術LUCIDを用いて検体を丸ごと透明化し、レーザー顕微鏡で観察することで、顕微鏡の解像度で検体の内部まで3次元情報として解析することができる。蛍光タンパク質や蛍光色素に対してほとんど褪色を示さないため、蛍光ラベルと併用して組織を観察することができる。透明化されたサンプルをレーザー顕微鏡で撮影すると、従来のカットする手法に換算して1000カットに相当する。得られる顕微鏡撮影像が従来の2,3カットに対して数百倍となるので、専門医が診断をするには負担が大きくなってしまう。したがって、人工知能が3次元画像を解析して病変を検知し専門医に提示することで病変の見落としリスク(false negative)がゼロになる診断支援システムを開発することが本研究の目的である。
