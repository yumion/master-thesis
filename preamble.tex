% jsbookで余白が広すぎるのを直す
% 参照 https://oku.edu.mie-u.ac.jp/~okumura/jsclasses/
\setlength{\textwidth}{\fullwidth}
\setlength{\evensidemargin}{\oddsidemargin}
\addtolength{\textwidth}{-5truemm}
\addtolength{\oddsidemargin}{5truemm}

% 同梱の ISEE 用の表紙テンプレ
\usepackage{thesis_cover}

% OTF フォントを使えるようにし、複数のウェイトも使用可能にする。
% これがないと、Mac のヒラギノ環境で使われる角ゴが太すぎてみっともない。
\usepackage[deluxe]{otf}
% OT1→T1に変更し、ウムラウトなどを PDF 出力で合成文字ではなくす
\usepackage[T1]{fontenc}
% uplatex の場合に必要な処理 
\usepackage[utf8]{inputenc} % エンコーディングが UTF8 であることを明示する。
\usepackage[prefernoncjk]{pxcjkcat} % アクセントつきラテン文字を欧文扱いにする
% Helvetica と Times を sf と rm のそれぞれで使う。
% default だとバランスが悪いので、日本語に合わせて文字の大きさを調整する。
\usepackage[scaled=1.05,helvratio=0.95]{newtxtext}
% 色
\usepackage[dvipdfmx]{color}
% 行番号を表示する。
\usepackage{lineno}

% latexdiff
% 実際の修論には入れる必要なし
%DIF PREAMBLE EXTENSION ADDED BY LATEXDIFF
%DIF UNDERLINE PREAMBLE %DIF PREAMBLE
\RequirePackage[normalem]{ulem} %DIF PREAMBLE
\RequirePackage{color}\definecolor{RED}{rgb}{1,0,0}\definecolor{BLUE}{rgb}{0,0,1} %DIF PREAMBLE
\providecommand{\DIFadd}[1]{{\protect\color{blue}\uwave{#1}}} %DIF PREAMBLE
\providecommand{\DIFdel}[1]{{\protect\color{red}\sout{#1}}}                      %DIF PREAMBLE
%DIF SAFE PREAMBLE %DIF PREAMBLE
\providecommand{\DIFaddbegin}{} %DIF PREAMBLE
\providecommand{\DIFaddend}{} %DIF PREAMBLE
\providecommand{\DIFdelbegin}{} %DIF PREAMBLE
\providecommand{\DIFdelend}{} %DIF PREAMBLE
%DIF FLOATSAFE PREAMBLE %DIF PREAMBLE
\providecommand{\DIFaddFL}[1]{\DIFadd{#1}} %DIF PREAMBLE
\providecommand{\DIFdelFL}[1]{\DIFdel{#1}} %DIF PREAMBLE
\providecommand{\DIFaddbeginFL}{} %DIF PREAMBLE
\providecommand{\DIFaddendFL}{} %DIF PREAMBLE
\providecommand{\DIFdelbeginFL}{} %DIF PREAMBLE
\providecommand{\DIFdelendFL}{} %DIF PREAMBLE
%DIF END PREAMBLE EXTENSION ADDED BY LATEXDIFF


%% 以下追加したpackage
% 画像の取り扱いに必要
\usepackage{graphicx}
% 数式の機能を拡張
\usepackage{amsmath}
\usepackage{bm}
\usepackage{upgreek}
\usepackage{newtxmath,newtxtext}
% コードを表示
\usepackage{listings}
\lstset{
    %プログラム言語(複数の言語に対応,C,C++も可)
    language = Python,
    %背景色と透過度
    backgroundcolor={\color[gray]{.98}},
    %枠外に行った時の自動改行
    breaklines = true,
    %自動改行後のインデント量(デフォルトでは20[pt])	
    breakindent = 10pt,
    %標準の書体
    basicstyle = \ttfamily\scriptsize,
    %コメントの書体
    commentstyle = {\itshape \color[cmyk]{1,0.4,1,0}},
    %関数名等の色の設定
    classoffset = 0,
    %キーワード(int, ifなど)の書体
    keywordstyle = {\bfseries \color[cmyk]{0,1,0,0}},
    %表示する文字の書体
    stringstyle = {\ttfamily \color[rgb]{0,0,1}},
    %枠 "t"は上に線を記載, "T"は上に二重線を記載
    %他オプション:leftline,topline,bottomline,lines,single,shadowbox
    frame = tbrl,
    %frameまでの間隔(行番号とプログラムの間)
    framesep = 5pt,
    %行番号の位置
    numbers = left,
    %行番号の間隔
    stepnumber = 1,
    %行番号の書体
    numberstyle = \tiny,
    %タブの大きさ
    tabsize = 4,
    %キャプションの場所("tb"ならば上下両方に記載)
    captionpos = t
}
% 複数引用
\usepackage{cite}
% jecon.bstを使う時に必要.それ以外ではエラーが出るのでコメントアウトする.
%\usepackage{natbib}
% 複数図を並べる時のcaption
\usepackage{subcaption}
\usepackage{here}
% 表関連
\usepackage{booktabs}
% 表でセルを複数列で結合する
\usepackage{multicol}
\usepackage{multirow}
% PDF 内で外部リンクや文書内リンクを生成したい場合に使う(好みによる)
\usepackage[dvipdfmx, hidelinks]{hyperref}
\usepackage{url}
% 複数行コメント
\usepackage{comment}
% 番号付き箇条書きのオプション利用
\usepackage{enumerate}
% renewcommandなどはここに
\usepackage{mymacros}
% 目次でsubsectionまで表示
\setcounter{tocdepth}{2}

