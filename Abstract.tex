% 170字以内
上肢機能障害者の生活を支援するロボットとして携帯可能な小型自律ロボット,パーソナルロボットハンドを開発した.試作1号機ではスマートフォンを用いたローカル演算に夜強化学習によって,特定の色の物体のピックアップに成功した.試作2号機では深層学習を用いて対象物を識別し,指定した物体に接近・把持し,元の場所へ持ち帰ることに成功した.
% (163)


% 最近の義手は主に筋電図(EMG)によって駆動される。しかしEMG駆動は汗に弱いという問題があり、トレーニングが必要である。一方、最近では、より人間の生活に密着したロボット、例えば、整体ロボットが開発されている。そこで本研究では、筋電図制御ではなく強化学習を用いた、人間の手とは独立したロボットハンドを提案する。今回、スマートフォンを使って特定の色の物体を拾うことに成功した。
