\chapter{試作1号機:スマホ搭載ハンド}
\newpage

\section{要求仕様}

環境を認識できるセンサ,対象物に接近するためのアクチュエータ,把持を行うアクチュエータ,これらを制御するコンピュータが必要である.
これらを考慮してロボットハンドにはスマホを搭載し,スマートフォンのフロントカメラから環境を認識する.またタイヤを搭載しDCモータで駆動させる.把持にはサーボモータを用いる.簡便性を考慮して強化学習の演算もスマホで行う.モータの制御にはArduinoを使用し,スマホとシリアル通信を行い強化学習によって出力される行動を実行に移す.


\section{機構設計・機体デザイン}


使用した部品を\tab{}まとめた.
スマートフォンHUAWEI P10 lite
DCギヤモーターhttp://akizukidenshi.com/catalog/g/gM-12379/
TAMIYA製スパイクタイヤ
GWSサーボ MICRO/2BBMG/FP(フタバ)
ArduinoProMini + FTDI


ロボットハンドのフレームは3DCAD(123D Design Autodesk社)で設計し,簡便さと軽さを考慮し3Dプリンタ(TITAN GENKEI社)で造形を行った.作製したロボットハンドの外観を示す(\fig{1号機外観}).スマホは腕の上部に装着し,フロントカメラに鏡を45度で置くことで前方を捉えることができる.また,腕部分にバッテリーとタイヤ,そしてArduinoを含む回路を収め,上から見るとスマホと手のみが見えるように工夫して組み立てた.


\section{制御アルゴリズム}
モータの制御を強化学習で行った.環境に対してロバストであるため実生活において有効だと考えた.
\fig{1号機アーキテクチャ}に1号機の制御系アーキテクチャを示す.

スマホのフロントカメラから写真を撮り,その画像をHSV空間に変換し赤色だけを抽出しターゲットのマスク画像を得る.マスク画像から面積(ピクセル数)と重心を求め,面積を画像サイズで規格化し0-100とした.この面積値と前フレームでの面積値との差分の2次元を状態として与えた.行動として前進,後退,右旋回,左旋回の4次元を与えた.報酬としてタスクが成功したら+1,1episode以内で成功できなかったら-1,その他では0とした.episodeの終了判定に面積値と重心座標を用いて,面積が40以上で最接近とし,また重心座標がロボットの中心座標から10pixel以内で正面に来たことを判定した.接近が完了したらサーボモータを動かして対象物を握るようルールベース化した.


\section{評価}
実機で2日間に渡り,約300episodes学習を行った.その学習の中で成功したepisodeにおける挙動を\fig{1号機例}にまとめた.まず,旋回することであたりを探索する.カメラにターゲットを捉えたらそれに向かって接近し,一定の距離まで近づいたら把持を行う.このような流れで学習が進んでいることが示唆された.

しかし,スマホではカメラのFPSが低く(1FPS程度)各stepにおける行動が1秒程度持続されてしまい,同じ場所を旋回しつづけるといった行動が助長になる問題があった.FPSが遅い原因として,スマホのカメラにプロテクトがかかっておりビデオが使えず,カメラの単写を使用しているためである.また,スマホの演算能力では,より大きな状態や行動次元は難しいといった課題がある.

\subsection{物理シミュレーション}


\section{まとめ}
スマートフォンを用いてlocalで完結したロボットハンドを作製した.
指定した色の物体を認識し,接近し,把持するという一連の動作を行うロボットハンドを開発し,赤色の物体に接近し把持することに成功した.より簡便性を高めるために環境を認識するセンサと,アクチュエータの制御にはスマホを用いて実装した.実世界での学習とシミュレーション環境での学習を両方行い,環境を観測することが学習に大きく関わることを示した.そして,接近のタスクにおいてはagent視点では学習が収束せず,agentとターゲットを共に俯瞰する視点を仮想的に考えた時の観測状態で正しく学習されることがわかった.

茨城県立大学付属病院リハビリテーション科にて義手使用患者へヒアリングを行い,ロボットハンド1号機を見せて率直な感想をいただいた.
「軽いと感じた」
「家に帰ったらテーブルの上で作業することが多いから,ロボットハンドの有用性はあると思う」
「自分のスマホでできるのが良い」
「クラウドを使わないから停電になったときでも使えて良い」
のように,義手使用患者にも有用性は確かにあると言える.

1号機の課題として以下が挙げられる.
機械的な自由度が少なく,様々な形状の物体を持ち上げて運ぶことが難しい.
様々な種類の物体を識別できない.
AndroidのスマホではPythonから制御すると動画が使えず,リアルタイム性に欠ける.
スマホの計算リソースでは重たい画像処理ができない.


