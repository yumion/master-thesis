\chapter{3D画像の半教師あり学習による,少量ラベルの腫瘍と正常の分類}

\section{胃がんの画像データ取得方法}

\subsection{透明化と染色}
ホルマリン固定されている検体をLUCIDによる透明化してから染色する.
多光子顕微鏡で3D画像として撮影する.一つの視野で観察することができる領域は縦横が~で深さ方向が~である.ここでLUCIDによる透明化によって今までは断面のみの撮影だったものを深さ方向まで撮影することができるようになった.
検体全ての撮影をするために画像取得にはタイリングをする.

\subsection{擬似HE染色}

\subsection{教師データの作成}

\section{古典的な画像処理手法による識別精度評価}
\subsection{2次元画像}
\subsection{3次元画像}

\section{教師あり学習による識別精度評価}
\subsection{2次元画像}
\subsection{3次元画像}

\section{教師なし学習による識別精度評価}
\subsection{2次元画像}
\subsection{3次元画像}

\section{半教師あり学習による識別精度評価}
\subsection{2次元画像}
\subsection{3次元画像}

