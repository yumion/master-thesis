\chapter{結論}
\newpage

\section{総括}

本論文では,実用的な電動義手の開発を目的として,3D プリントと jamming 転移現象を 活用した義手試作を行った.jamming グリッパのモデル化と,Bio-mimic 吸盤の開発に取り 組み,実際に吸盤構造体を搭載することで,従来の jamming グリッパでは困難とされてい る扁形物体の把持が可能になった.これにより jamming グリッパの能力を向上・拡張する ことに成功した.
義手の試作にあたっては,実際に東京大学医学部付属病院にて実際の義手利用者からヒ アリングを行い,義手の外観が利用者の心理的ならびに QOL の観点から非常に大きなウェ イトを占めているとの知見を得た.そこで,著者自身のリアルな手をモデルにするととも に,人間の手が行う把持機能に注目してプロトタイプを作製した.作製した電動義手は, ローコスト,軽量,耐水性といった特徴を持つとともに,握力把持,精密把持の 2 種類の 方式での把持能力を持っていることを確認した.

本研究ではパーソナルロボットを小型化し,義手使用患者や片麻痺患者など上肢機能障害患者を対象としたパーソナルロボットを開発する.義手は常に身につけているように,パーソナルロボットも携帯できるよう小型化し腕の形をしたロボットハンドとする.上肢機能障害者にとってはパーソナルロボットはどんな事態でも動作する必要があるため,今回はCloudは使用せずEdgeで処理を行うこととする.また座って机で作業することが多いため使用場所を机の上に限定し,指定した物をピックアップするタスクを実行できるパーソナルロボットハンドを作製することを目的とする.





本研究では,上肢機能障害者支援を目指し,強化学習・深層学習をロボット制御に適用した携帯可能な自律型ロボットハンドの試作を行った.

\subsection*{試作1号機}



\subsection*{試作2号機}



\section{今後の展望}
\subsection*{デザインの改善}
地面から直立している物体のみで,平たい物体や細長く自立しない物体などは把持不可能であった.そのため,腕の関節と手首の関節を増やすことで,地面に対して垂直にアプローチできるように改善すると把持できる物体の幅が広がる.また,今回グリッパは大した考慮をしていなかったため,様々な物体を把持可能であるJammingグリッパ\cite{jamminggripper}など,グリッパ形状を検討する必要がある.

\subsection*{ハードウェア(計算リソース)の改善}
2号機では識別能力を向上させるために,外部の計算リソースを使用していたためGPU搭載PCと有線で接続されていたことが大きな課題であった.そこで次世代機ではEdgeデバイスであるJetsonNanoを使用することで,localでもGPUを使用した推論が可能となる.さらにJetsonNanoはanalogWriteができるため,Arduinoを必要とせずアクチュエータの制御が可能となり,よりシンプルな設計にできると考えられる.

\subsection*{ソフトウェアの改善}
接近タスクは比例制御で十分有効であることがわかったが,把持タスクはルールベースでは物体の形状に対称性が無いものは把持成功率が低かった.把持タスクにはVisionベースの教師あり学習を行い最適な把持点を掴むことで改善できる.さらに識別タスクに用いたMask R-CNNと合わせてモデルを構築できれば,End-to-Endに学習ができ精度改善や推論速度向上を期待できる.
