\chapter{自分の研究本体を述べるところ}
ここは自分のやった研究を述べる章です。実際の中身に合わせて章を複数立てにする場合もあると思います。「議論」の章を別に分ける場合は、この章では得られた結果までを記述し、その結果に対する議論は「議論」の章に回すのが良いでしょう。この章は必ずしも1つの章のみである必要はありません。研究内容に応じて、複数の章に分割することも一般的に行われます。

修士論文で大切なことは、第~\ref{chap_intro}~章や第~\ref{chap_review}~章で述べた伏線(研究の目的と動機)を回収するべく、きちんと研究内容を順序立てて書き、また自分の貢献を明確にすることです。論文全体で論理展開がきちんとしていれば良いので、必ずしも実際に行った実験などの時系列でこの章を書き進める必要はありません。また修士論文としての完成度が大切ですので、修士論文のテーマに直接関係のない自分のやったことを無理に混ぜる必要もありません。
