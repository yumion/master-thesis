\chapter{病理画像の診断を支援する機械学習アルゴリズム}
\label{chap_review}

この章では、自分の研究に関連する分野の歴史や現状について説明したり、研究を展開する上で重要となる知識の解説を行います。\footnote{省略語は必ず正式名称を先に書き、省略系は丸括弧に入れます。省略語はあくまで「以降このように略す」という用途だからです。また、日本語文章中で使う丸括弧は()ではなく()です。人工知能(Artificial Intelligence: AI)}
病理画像をデジタルで保存することが始まったのは数十年前になる.これによって遠隔地でも診断することができるようになったり,情報を共有することができるようになり,複数の医師で診断しミスを防止するsecond opinionが容易になった.計算機科学の分野の側面ではデータを収集することができるようになり,研究が盛んに行われることになった.その後は,様々な病理データでより改良されたアルゴリズムの提案が行われている.

