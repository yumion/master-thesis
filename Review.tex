\chapter{深層学習による病理画像の診断支援}
\label{chap_review}
病理画像をデジタルで保存することが始まったのは数十年前になる.これによって遠隔地でも診断することができるようになったり,情報を共有することができるようになり,複数の医師で診断しミスを防止するsecond opinionが容易になった.計算機科学の分野の側面ではデータを収集することができるようになり,研究が盛んに行われることになった.その後は,様々な病理データでより改良されたアルゴリズムの提案が行われている.

細胞組織の形態を観察するための病理染色ではヘマトキシリン・エオジン染色(HE染色)が一般的に用いられる.細胞核を青紫色に染色し,細胞質をピンク色に染色する.正常から異常に変化していくと,細胞核が過度に増殖したり,細胞質の形が崩れたりすることで,その特徴を機械学習によって精度よく検出するための研究が行われている.

これまでは,核の形やテキスチャーからパターンマッチングなどの画像処理によって腫瘍を検出する研究されてきたが,近年になって画像処理に大きなブレークスルーが起きたことをきっかけに,新しい手法で解析するようになってきた.そのブレークスルーがディープラーニングである.

\section{ニューラルネットワーク}
\subsection*{多層パーセプトロン}

\section{画像認識におけるディープラーニング}
ディープラーニングが注目を受けることになったのは、2012の画像認識コンペティションであるILSVCでHintonがAlexNetというDeepLearningの手法で他のチームを圧倒する精度を出したことが始まりである.

ディープラーニングは、その名の通り、今までのニューラルネットワークよりも層が深くなっている。この層の深さが、より複雑な特徴を抽出することができる。層の深さはモデル設計でチューニングしていく必要があり、上述のAlexNetは8層のネットワークであった。しかし、ネットワークの層は深くなっていき、現在では152層のResNetというものがILSVC2015でもっとも精度が良く、人間の認識精度を超えている。ディープラーニングによる学習方法について、畳み込みニューラルネットワーク(Convolutional Neural Network : CNN)を利用する。入力する画像に対して小さなサイズ(3×3など)のフィルターを畳み込み計算をする。このフィルターの値を学習することがディープラーニングの学習である。フィルターの数も設計する必要がある。学習については、誤差逆伝播法を利用する。ディープラーニングの出力結果と正解との比較をした時に、どれだけ正解から離れているかを評価する誤差関数を使って、その誤差関数が小さくなるように、フィルターの重みを変更する時に勾配降下法を使う。


画像処理におけるディープラーニングができることは、画像から(1)カテゴリ分けをすること、(2)物体を検出すること、(3)物体のセグメンテーションをすることである.

\subsection*{物体検出}

\subsection*{セグメンテーション}

\subsection*{深層学習による3次元画像解析}

\section{教師なし学習}

\section{半教師あり学習}

\section{ディープラーニングを用いた病理画像診断}
学習した結果を評価するときに医療画像の場合は、PrecisionとRecallを識別性能の評価指数として使う。

\begin{align}
  \mathrm{Precision} & = \dfrac{TP}{TP+FP}\\
  \mathrm{Recall} & = \frac{TP}{TP+FN}
\end{align}

ここで使ったTP(True Positive)は真の結果が正である時に予測も正であるという意味であるTN(True Negative)は真の結果が正で予測が負である場合である。同様にFP(False Positive)真の結果が負で予測も負、FN(False Negative)真の結果が負で予測は正である.

2016年に行われたCAMELYON16という乳がんの転移を調べるシステムのコンペティションでは、7人の医師の成績を上回っていることが報告されてている。

また皮膚癌の種類が757種類を細かい違いまで識別するように学習する。学習方法はInception v3というネットワークを利用している。これは22層の畳み込みニューラルネットワークである。またこのネットワークは皮膚癌の識別タスクを学習する前に、IMageNetというILSVCでも使っている一般物体画像を使って事前にネットワークの重みの初期値を決めることによって、精度が向上させる。これを転移学習(Transfer Learning)と呼ばれている。
