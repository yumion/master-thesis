\chapter{病理画像の診断を支援する機械学習}
\label{chap_review}
病理画像をデジタルで保存することが始まったのは数十年前になる.これによって遠隔地でも診断することができるようになったり,情報を共有することができるようになり,複数の医師で診断しミスを防止するsecond opinionが容易になった.計算機科学の分野の側面ではデータを収集することができるようになり,研究が盛んに行われることになった.その後は,様々な病理データでより改良されたアルゴリズムの提案が行われている.

細胞組織の形態を観察するための病理染色ではヘマトキシリン・エオジン染色(HE染色)が一般的に用いられる.細胞核を青紫色に染色し,細胞質をピンク色に染色する.正常から異常に変化していくと,細胞核が過度に増殖したり,細胞質の形が崩れたりすることで,その特徴を機械学習によって精度よく検出するための研究が行われている.

これまでは,核の形やテキスチャーからパターンマッチングなどの画像処理によって腫瘍を検出する研究されてきたが,近年になって画像処理に大きなブレークスルーが起きたことをきっかけに,新しい手法で解析するようになってきた.そのブレークスルーがディープラーニングである.

\section{ディープラーニング}
機械学習の手法で病理画像を診断する手法では,多くはセグメンテーションと物体検出が使われてきた.