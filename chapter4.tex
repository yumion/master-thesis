\chapter{試作2号機:インスタンス認識ハンド}
\newpage

\section{要求仕様}
1号機の課題の中で,特に自由度が少ない点と物体を識別できない点は日常生活で使用する上で必須である.そこで2号機ではこれら2点の課題を克服したロボットハンドを開発した.



\section{機構設計・機体デザイン}



\section{制御アルゴリズム}


\section{評価}


\section{まとめ}




