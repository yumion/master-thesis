\chapter{ロボットハンドの開発}
\label{chap_result}

\section{試作1号機:スマートフォン搭載型ハンド}

\subsection{要求仕様}
本研究の目的を達成するために,環境を認識できるセンサ,対象物に接近するためのアクチュエータ,把持を行うアクチュエータ,これらを制御するコンピュータが必要である.
これらを考慮してロボットハンドにはスマホを搭載し,スマートフォンのフロントカメラから環境を認識する.またタイヤを搭載しDCモータで駆動させる.把持にはサーボモータを用いる.簡便性を考慮して強化学習の演算もスマホで行う.モータの制御にはArduinoを使用し,スマホとシリアル通信を行い強化学習によって出力される行動を実行に移す.


\subsection{機構設計・機体デザイン}

使用した部品を次にまとめた.
スマートフォンHUAWEI P10 lite
DCギヤモーターhttp://akizukidenshi.com/catalog/g/gM-12379/
TAMIYA製スパイクタイヤ
GWSサーボ MICRO/2BBMG/FP(フタバ)
ArduinoProMini + FTDI


ロボットハンドのフレームは3DCAD(123D Design Autodesk社)で設計し,簡便さと軽さを考慮し3Dプリンタ(TITAN Genkei社)で造形を行った.作製したロボットハンドの外観を示す(\fig{}).スマホは腕の上部に装着し,フロントカメラに鏡を45度で置くことで前方を捉えることができる.また,腕部分にバッテリーとタイヤ,そしてArduinoを含む回路を収め,上から見るとスマホと手のみが見えるように工夫して組み立てた.


\subsection{制御系}
モータの制御を強化学習で行った.
環境に対してロバストであるため実生活において有効だと考えた.


\subsection{評価}


\subsection{まとめ}
スマートフォンを用いてlocalで完結したロボットハンドを作製した.


茨城県立大学付属病院リハビリテーション科にて義手使用患者へヒアリングを行い,ロボットハンド1号機を見せて率直な感想をいただいた.
「軽いと感じた」
「家に帰ったらテーブルの上で作業することが多いから,ロボットハンドの有用性はあると思う」
「自分のスマホでできるのが良い」
「クラウドを使わないから停電になったときでも使えて良い」
のように,義手使用患者にも有用性は確かにあると言える.

1号機の課題として以下が挙げられる.
機械的な自由度が少なく,様々な形状の物体を持ち上げて運ぶことが難しい.
様々な種類の物体を識別できない.
Pythonから制御すると動画が使えず,リアルタイム性に欠ける.
スマートフォンの計算リソースでは重たい画像処理ができない.


\section{試作2号機:インスタンス認識ハンド}
1号機の課題の中で,特に自由度が少ない点と物体を識別できない点は日常生活で使用する上で必須である.そこで2号機ではこれら2点の課題を克服したロボットハンドを開発した.

\subsection{要求仕様}


\subsection{機構設計・機体デザイン}



\subsection{制御系}


\subsection{評価}


\subsection{まとめ}




