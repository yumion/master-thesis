\chapter{結果}\label{chap_result}
%loss,accのグラフは全て.
%前処理を行った場合は,その画像の一例も載せる
%モデルの画像(appendixの方がいいかもしれないです)
%各sectionで異なる場合は各sectionで載せましょう
%なるべくpdfで保存しましょう

\section{古典的な画像処理手法による識別精度評価}
%楕円検出の塗った画像
%loss,accのグラフ


\section{教師あり学習による識別精度評価}
% 実際のラベルの画像
% loss, accグラフ



\section{教師なし学習による識別精度評価}
% 潜在空間の分布
% 写真での分布
% loss, accグラフ
\subsection{VAE}
擬似HE染色した画像と元のカラー画像のそれぞれに対してVAEを行い,潜在変数を2次元空間にプロットした結果を\fig {VAEplot}に示す.
\begin{figure}[H]
	\centering
	
	\begin{minipage}[b]{0.45\columnwidth}
		\centering
		\includegraphics[clip, width=\linewidth]{fig/variational_auto_encoder/vae_colon_epoch_100_c13_he}
		\subcaption{HE like color at sample A}
		\label{fig:}
	\end{minipage}
	\begin{minipage}[b]{0.45\columnwidth}
		\centering
		\includegraphics[clip, width=\linewidth]{fig/variational_auto_encoder/vae_colon_epoch_299_c13_rgb}
		\subcaption{original color at sample A}
		\label{fig:}
	\end{minipage}
	\begin{minipage}[b]{0.45\columnwidth}
		\centering
		\includegraphics[clip, width=\linewidth]{fig/variational_auto_encoder/vae_colon_epoch_100_he_mix}
		\subcaption{HE like color at sample A and B}
		\label{fig:}
	\end{minipage}
	\begin{minipage}[b]{0.45\columnwidth}
		\centering
		\includegraphics[clip, width=\linewidth]{fig/variational_auto_encoder/vae_colon_epoch_100_rgb_mix}
		\subcaption{original color at sample A and B}
		\label{fig:}
	\end{minipage}
	
	\caption{Latent space of 2D. Color ratio 1 is cancer, 0 is normal.}
	\label{fig:VAEplot}

\end{figure}

\subsection{GAN}

\begin{figure}[H]
	\centering
	
	\begin{minipage}{0.24\columnwidth}
		\centering
	    \includegraphics[clip, width=\linewidth]{fig/generative_adversarial_nets/0000_0000}
		\subcaption{epochs = 0}
		\label{fig:}
	\end{minipage}
	\begin{minipage}{0.24\columnwidth}
		\centering
		\includegraphics[clip, width=\linewidth]{fig/generative_adversarial_nets/0079_0000}
		\subcaption{epochs = 79}
		\label{fig:}
	\end{minipage}
	\begin{minipage}{0.24\columnwidth}
		\centering
		\includegraphics[clip, width=\linewidth]{fig/generative_adversarial_nets/0641_0000}
	\subcaption{epochs = 641}
	\label{fig:}
	\end{minipage}
	\begin{minipage}{0.24\columnwidth}
		\centering
		\includegraphics[clip, width=\linewidth]{fig/generative_adversarial_nets/0969_0000}
	\subcaption{epochs = 969}
	\label{fig:}
	\end{minipage}
	\begin{minipage}{0.24\columnwidth}
		\centering
		\includegraphics[clip, width=\linewidth]{fig/generative_adversarial_nets/1213_0000}
	\subcaption{epochs = 1213}
	\label{fig:}
	\end{minipage}
	\begin{minipage}{0.24\columnwidth}
		\centering
		\includegraphics[clip, width=\linewidth]{fig/generative_adversarial_nets/1619_0000}
		\subcaption{epochs = 1619}
		\label{fig:}
	\end{minipage}
	\begin{minipage}{0.24\columnwidth}
		\centering
		\includegraphics[clip, width=\linewidth]{fig/generative_adversarial_nets/2004_0000}
		\subcaption{epochs = 2004}
		\label{fig:}
	\end{minipage}
	\begin{minipage}{0.24\columnwidth}
		\centering
		\includegraphics[clip, width=\linewidth]{fig/generative_adversarial_nets/3208_0000}
		\subcaption{epochs = 3208}
		\label{fig:}
	\end{minipage}
	
	\caption{Transition generated images by GAN}
	\label{fig:GANimage}

\end{figure}


\section{半教師あり学習による識別精度評価}

%lossとaccグラフ
% 教師ありと半教師ありを同じグラフにaccをプロット
