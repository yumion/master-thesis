\chapter*{謝辞}%
\addcontentsline{toc}{chapter}{謝辞}%

「謝辞」(acknowledgments)は、修士論文を作成する上であなたを支えてくれた人への感謝を書く場所です。誰かへの感謝の気持ちを公開の場所で文書にするというのは気恥ずかしいものですし、修士論文以外ではそんなことをした経験がないかもしれませんが、投稿論文では一般的に行われます\footnote{ただし、投稿論文では家族や友人への感謝はあまり書かず、研究費を出した機関や研究の協力をしてくれた研究者などを書くことが多いです。}。

多くの修士論文では指導教員(国立大学の法人化後は、指導教「官」とは言いません)、実験協力してくれた共同研究者、間接的に助言などをくれた研究室の他の教員・先輩・同輩・後輩、研究室の秘書さんなどに謝意を示すことが多いようです。もし奨学金をどこからか受給していたら、奨学金の出所に対しても謝辞を書いても良いでしょう。また家族・恋人・友人に対する感謝も見られますが、恋人の名前は将来隠したくなる場合もあるので注意しましょう。

感謝の気持ちを書く場所ですので、その相手に失礼のないようにしましょう。氏名の漢字の間違いや、職階の間違いが頻繁に見られます。特に助教を助教授と書き間違えたり、准教授を助教授としたりという間違いが目立ちます。


小野寺先生,武田さん,山田くん,樽茶さん,鳴海さん,染谷先生,小野先生,水谷先生,長沼先生,原田先生,福田さん,鷹野さん,田中さん,藤平さん,ありがとう.

