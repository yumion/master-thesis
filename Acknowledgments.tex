\chapter{謝辞}
本研究は,東京大学大学院工学系研究科電気系工学専攻小野寺宏特任教授のご指導の下で行われました.この2年間研究を進めるにあたり,研究室内外の多くの方に多大なるご協力をいただきました.この場を借りて謝辞を述べさせていただきます.

小野寺宏 特任教授\\
指導教員として2年間ご指導いただきました.充実した研究環境を与えていただき,また幅広い先生のご紹介をしてくださり研究の異分野連携の重要性を知ることができました.毎週研究の進捗の報告でご指導いたたき,研究に行き詰まってもアイディアをたくさんいただき,研究を進めることができました.研究だけでなく今後の進路についてや医療業界の知識などを普段から教えていただき人生の選択肢を広げることができました.大変ありがとうございました.

染谷隆夫 教授\\
研究の進捗の発表の場を定期的に設けていただき,研究で困っていた部分や疑問に思うところについて染谷研究室の学生を含め,ディスカッションをすることができ研究を前に進めることができました.大変ありがとうございました.

横田和之 准教授\\
研究の方向性を心配くださりありがとうございました.また,実験TAでは電子回路の深い知識と考察や実験の注意点を丁寧に教えていただきました.大変ありがとうございました.

九州工業大学 長隆之 准教授\\
強化学習のノウハウについて丁寧に一から教えていただきました.東大所属でなくなってもメールでのやりとりを続けさせて頂き,とても助かりました.大変ありがとうございました.

\newpage
茨城県立医療大学付属病院 四津有人 准教授\\
リハビリテーション科の見学を快諾してくださいました.ヒアリングをさせて頂き,研究のモチベーションを再確認できました.大変ありがとうございました.

武田伊織 博士\\
3Dプリンターの使い方を丁寧に教えていただいたり,日々の研究で困っている時にアイディアをいただきました.学会発表や研究発表の際に発表資料の作成をたくさんご指導していただきました.大変ありがとうございました.

松崎博貴 氏\\
深層学習に関するプログラミングについてアドバイスいただきました.食事をよく共にして,本郷周りの美味しい飯屋を教えていただきました.またベンチャーやがんセンターの紹介等,プライベートでもたくさん道を開いてくださって感謝しています.大変ありがとうございました.

多川友作 氏\\
研究において活発な議論をさせていただきました.また筋トレに付き合っていただきました.大変ありがとうございました.

竹内雅樹 氏\\
C言語周りのアドバイスをいただきました.大変ありがとうございました.来年の修論頑張ってください.

矢谷研究室 チェ シウク 氏\\
アルゴリズム面でクリティカルなアドバイスをいただきました.崔さんのおかげで問題点がいくつも解決できました.大変ありがとうございました.

鷹野玲美 学術支援専門職員\\
電子工作に関してたくさんご指導いただきました.また一緒にSFPに参加し\href{https://qiita.com/yumion/items/b7fa89f29504cab1f123}{YUBIBO}を完成させ,テックコンテストに出場したこと,とても嬉しかったです.大変ありがとうございました.

田中麻美 技術員\\
普段からの雑談で研究をする元気を与えてくれました.いつも田中さんの笑顔に癒されていました.大変ありがとうございました.

\vspace{12pt}
両親や友人には,健康に気を使ってくれたり様々な面で支えていただきました.ありがとうございました.
