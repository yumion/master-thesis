\chapter*{謝辞}%
\addcontentsline{toc}{chapter}{謝辞}%

小野寺宏特任教授
指導教員として2年間ご指導いただきました.充実した研究環境を与えていただき,また幅広い先生のご紹介をしてくださり研究の異分野連携の重要性を知ることができました.毎週研究の進捗の報告でご指導いたたき,研究に行き詰まってもアイディアをたくさんいただき,研究を進めることができました.研究だけでなく今後の進路についてや医療業界の知識などを普段から教えていただき人生の選択肢を広げることができました.大変ありがとうございました.

染谷隆夫教授
研究の進捗の発表の場を定期的に設けていただき,研究で困っていた部分や疑問に思うところについて染谷研究室の学生を含め,ディスカッションをすることができ,研究を前に進めることができました.大変ありがとうございました.

横田和之先生
実験TAでは,電子回路の深い知識と考察や,実験の注意点を丁寧に教えていただきました.また機械学習についてのディスカッションをして理解を深めるきっかけとなりました.大変ありがとうございました.

小野敏嗣助教
内視鏡生検の検体をいただき,検体の特徴だけでなく,内視鏡生検の知識や経験を教えていただき解析方法の検討に役立てることができました.また実際の内視鏡生検に立ち会わせていただき,現場の様子を見ることで,どのように研究成果を利用していくかを構想することができました.大変ありがとうございました.

長沼和則 特任研究員
本研究の標本処理システムの構築についてアドバイスをいただき,研究を進めることができました.大変ありがとうございました.

添田 建太郎 特任研究員
本研究の生検検体の染色と透明化を自動処理するためのマイクロチップ作成において,とても細かい造形にこだわって作ってくださりました.大変ありがとうございました.

原田達也 教授
画像処理の知識をご教授いただき,医療画像の解析データについて,人が見やすい画像にすることは機械学習にとっても判断しやすくなるという助言をいただき,その後の研究に役立てることができました.大変ありがとうございました.

牛久祥孝 講師
本研究の画像取得方法が従来のHE染色方法とは異なるため,解析に困っていたころ,色空間を変換して擬似的なHE染色の画像に変換したことが,病理医にとって見やすいとの助言をいただき,これで機械学習でも処理することが重要であると分かりました.大変ありがとうございました.

武田伊織 特任研究員
3Dプリンターの使い方を丁寧に教えていただいたり,日々の研究で困っている時にアイディアをいただきました.また研究以外でも昼食に出かけて大学生活の悩みをスッキリさせることができました.発表資料の作成については,学会発表や研究発表の際に,たくさんご指導していただきました.大変ありがとうございました.

樽茶好彦 様
修士1年の際に,研究室の先輩として研究の相談だけでなく,修士過程の過ごし方や研究の進め方,就職についてご指導いただきました.大変ありがとうございました.

山田敦史 君
研究や大学生活において辛い時期でも相談に乗ってくれ,研究へのモチベーションを維持することができました.また研究テーマでお互いに機械学習を利用していたため,遅い時間まで研究のディスカッションに付き合ってくれました.大変ありがとうございました.

水谷浩哉 様
消化器内科の基本的な知識を教えていただきました.検体の説明を丁寧にしてくださり,本研究の解析データを正しく理解することができました.大変ありがとうございました.

福田圭佑 様
機械学習のプログラミングを丁寧に教えていただきました.また困った時には,アイディアからコードのレビューまで相談に乗ってくれて,研究を進めることができました.大変ありがとうございました.

鷹野玲美 学術支援専門職員
研究で必要な解析や画像取得などで,作業のお願いをしてから実行までがとても早く大変助かりました.また日々の研究生活で気遣ってくださり,精神的にも支えてくださりました.大変ありがとうございました.

田中麻美 技術員
研究の相談に乗ってくれたり,普段からの雑談で研究をする元気を与えてくれました.大変ありがとうございました.

藤平 まなみ 技術員
擬似HEの画像変換など,研究の処理で必要になった作業などをお願いしてお手伝いいただきました.大変ありがとうございました
.
両親には,健康に気を使ってくれたり様々な面で支えていただきました.ありがとうございました.
