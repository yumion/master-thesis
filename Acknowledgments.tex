\chapter*{謝辞}%
\addcontentsline{toc}{chapter}{謝辞}%

小野寺宏 特任教授\\
指導教員として2年間ご指導いただきました.充実した研究環境を与えていただき,また幅広い先生のご紹介をしてくださり研究の異分野連携の重要性を知ることができました.毎週研究の進捗の報告でご指導いたたき,研究に行き詰まってもアイディアをたくさんいただき,研究を進めることができました.研究だけでなく今後の進路についてや医療業界の知識などを普段から教えていただき人生の選択肢を広げることができました.大変ありがとうございました.

染谷隆夫 教授\\
% 研究の進捗の発表の場を定期的に設けていただき,研究で困っていた部分や疑問に思うところについて染谷研究室の学生を含め,ディスカッションをすることができ,研究を前に進めることができました.大変ありがとうございました.

横田和之 准教授\\
% 実験TAでは,電子回路の深い知識と考察や実験の注意点を丁寧に教えていただきました.また機械学習についてのディスカッションをして理解を深めるきっかけとなりました.大変ありがとうございました.

九州工業大学 長隆之 准教授\\


茨城県立大学付属病院 四津有人 准教授\\


武田伊織 特任研究員\\
3Dプリンターの使い方を丁寧に教えていただいたり,日々の研究で困っている時にアイディアをいただきました.また研究以外でも昼食に出かけて大学生活の悩みを解消することができました.発表資料の作成については学会発表や研究発表の際にたくさんご指導していただきました.大変ありがとうございました.

松崎博貴 氏\\


多川友作 氏\\


竹内雅樹 氏\\


鷹野玲美 学術支援専門職員\\
研究で必要な解析や画像取得などで,作業のお願いをしてから実行までがとても早く大変助かりました.また日々の研究生活で気遣ってくださり,精神的にも支えてくださりました.大変ありがとうございました.

田中麻美 技術員\\
研究の相談に乗ってくださり,普段からの雑談で研究をする元気を与えてくれました.大変ありがとうございました.

\vspace{12pt}
両親や友人には,健康に気を使ってくれたり様々な面で支えていただきました.ありがとうございました.
