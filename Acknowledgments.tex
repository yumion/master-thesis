\chapter*{謝辞}%
\addcontentsline{toc}{chapter}{謝辞}%

小野寺宏特任教授
指導教員として2年間,ご指導いただきました.大変ありがとうございました.

染谷隆夫教授
研究の進捗の発表の場を定期的に設けていただき,研究で困っていた部分や疑問に思うところについて染谷研究室の学生を含め,ディスカッションをすることができ,研究を前に進めることができました.大変ありがとうございました.

横田和之先生
実験TAでは,電子回路の深い知識と考察や,実験の注意点を丁寧に教えていただきました.また機械学習についてのディスカッションをして理解を深めるきっかけとなりました.大変ありがとうございました.

小野敏嗣助教
内視鏡生検の検体をいただき,検体の特徴だけでなく,内視鏡生検の知識や経験を教えていただき解析方法の検討に役立てることができました.また実際の内視鏡生検に立ち会わせていただき,現場の様子を見ることで,どのように研究成果を利用していくかを構想することができました.大変ありがとうございました.

長沼和則 特任研究員
本研究の標本処理システムの構築についてアドバイスをいただき,研究を進めることができました.大変ありがとうございました.

添田 建太郎 特任研究員
本研究の生検検体の染色と透明化を自動処理するためのマイクロチップ作成において,とても細かい造形にこだわって作ってくださりました.大変ありがとうございました.

原田達也 教授
画像処理の知識をご教授いただき,医療画像の解析データについて,人が見やすい画像にすることは機械学習にとっても判断しやすくなるという助言をいただき,その後の研究に役立てることができました.大変ありがとうございました.

牛久祥孝 講師
本研究の画像取得方法が従来のHE染色方法とは異なるため,解析に困っていたころ,色空間を変換して擬似的なHE染色の画像に変換したことが,病理医にとって見やすいとの助言をいただき,これで機械学習でも処理することが重要であると分かりました.大変ありがとうございました.

武田伊織 特任研究員

樽茶好彦さん

山田敦史くん

水谷浩哉さん
消化器内科の基本的な知識を教えていただきました.検体の説明を丁寧にしてくださり,本研究の解析データを正しく理解することができました.大変ありがとうございました.

福田圭佑さん
機械学習のプログラミングを丁寧に教えていただきました.また困った時には,アイディアからコードのレビューまで相談に乗ってくれて,研究を進めることができました.大変ありがとうございました.

鷹野玲美さん

田中麻美さん

藤平 まなみさん